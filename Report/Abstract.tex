\section*{ABSTRACT}
% This project focuses on the comprehensive simulation and performance analysis of an Orthogonal Frequency Division Multiplexing (OFDM) communication system enhanced with channel coding, Linear Minimum Mean Square Error (LMSE) equalization, and 64 Quadrature Amplitude Modulation (64 QAM). OFDM, a widely used modulation technique, divides the frequency spectrum into orthogonal subcarriers, making it resilient to channel impairments.

% The project's primary objectives include the implementation of channel coding techniques, such as Reed-Solomon or convolutional coding, to introduce redundancy for error correction. Additionally, LMSE equalization is employed to mitigate the impact of channel distortions, enhancing the overall system robustness.

% The modulation aspect of the project involves the utilization of 64 QAM, a higher-order modulation scheme, allowing for increased data rates. The combination of these techniques aims to optimize the system's spectral efficiency and error resilience in the presence of noise, interference, and fading channels.

% The simulation workflow encompasses the generation of random binary data, channel coding, 64 QAM modulation, OFDM modulation, channel modeling, additive white Gaussian noise (AWGN) introduction, LMSE equalization, and subsequent demodulation. The Bit Error Rate (BER) or Symbol Error Rate (SER) is then estimated to quantify the system's performance.

% The results of this simulation will provide valuable insights into the effectiveness of the integrated components in enhancing the reliability and efficiency of the OFDM communication system. The project's findings have implications for the design and optimization of communication systems in various wireless and broadband applications.

Modern communication systems face challenges of inter-symbol interference (ISI) and inter-carrier interference (ICI), particularly in single carrier systems with long symbol periods or multicarrier systems with closely spaced carriers. Orthogonal Frequency Division Multiplexing (OFDM) emerges as a solution by combining numerous low data rate carriers to construct a high data rate communication system. The orthogonality of carriers in OFDM enables close spacing without ICI, addressing the limitations of traditional multicarrier systems.

This project aims to demonstrate and explore the feasibility of an OFDM system, investigating the impact of key parameters on its performance. A Python program is developed for simulating a basic OFDM system, providing insights into its mechanisms and characteristics.
% The report is structured to offer a theoretical background in Chapter 1, essential for understanding the project's concepts. Chapter 2 details the design and implementation of the OFDM system, while Chapter 3 presents the simulation results and performance evaluation.

The comprehensive study of OFDM presented in this project contributes to the understanding of its advantages and limitations, laying the foundation for further research and applications in the evolving landscape of wireless communication.
\newpage
\cleardoublepage
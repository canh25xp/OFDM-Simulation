\section*{CHAPTER 2:  DESIGN AND IMPLEMENTATION}
\addcontentsline{toc}{section}{\numberline{}CHAPTER 2: DESIGN AND IMPLEMENTATION}
\setcounter{section}{2}
\setcounter{subsection}{0}
\setcounter{figure}{0}
\setcounter{table}{0}

\subsection{OFDM system architechure}
Figure \ref{diagram} shows a block diagram of our generic OFDM system.

\begin{figure}[htbp]
    \centering
    \documentclass[preview]{standalone}

\usepackage[english]{babel}
\usepackage{amsmath}
\usepackage{amssymb}
\usepackage[f]{esvect}
\usepackage{tikz}

\begin{document}

\usetikzlibrary{positioning}
\tikzset{block/.style={draw,thick,minimum width=1cm,minimum height=1cm,align=center}}
\tikzset{node distance=0.5cm}
\tikzset{double distance=1pt}
\tikzset{>=latex}

\begin{tikzpicture}
    \begin{scope}
        \draw [->] (-0.5,0) node [left] {$\vec{b}$}  -- (0,0) node (SP1) [right,block] {S2P};
        \node (M) [block,right=of SP1] {Mapping};
        \node (IFFT) [block,right=of M] {IFFT};
        \node (PS1) [block,right=of IFFT] {P2S};
        \node (CP) [block,right=of PS1] {Add CP};

        \draw [double,->] (PS1) -- (CP);

        \def\lines{
        \draw \ls ([yshift=3.mm]\from.east) -- ([yshift=3.mm]\to.west);
        \draw \ls ([yshift=1.mm]\from.east) -- ([yshift=1.mm]\to.west);
        \draw \ls ([yshift=-1.mm]\from.east) -- ([yshift=-1.mm]\to.west);
        \draw \ls ([yshift=-3.mm]\from.east) -- ([yshift=-3.mm]\to.west);
        }

        \def\ls{[->]}
        \def\from{SP1} \def\to{M} \lines
        \def\ls{[->,double]}
        \def\from{M} \def\to{IFFT} \lines
        \def\from{IFFT} \def\to{PS1} \lines

        \node (C) [block,below right=of CP] {Channel};

        \node (CP1) [block,below left=of C] {Remove CP};
        \node (SP2) [block,left=of CP1] {S2P};
        \node (FFT) [block,left=of SP2] {FFT};
        \node (EQ) [block,left=of FFT] {Equalize};
        \node (CE) [block,below=of EQ] {Channel\\Estimate};
        \node (Dem) [block,left=of EQ] {Demapping};
        \node (PS2) [block,left=of Dem] {P2S};
        \draw [->] (PS2.west) -- +(-0.5,0) node [left] {$\hat{b}$};

        \def\ls{[<-,double]}
        \def\from{SP2} \def\to{CP1} \lines
        \def\from{FFT} \def\to{SP2} \lines
        \def\from{EQ} \def\to{FFT} \lines
        \def\from{Dem} \def\to{EQ} \lines
        \def\ls{[<-]}
        \def\from{PS2} \def\to{Dem} \lines

        \draw [->,double,thick] (FFT.south) |- (CE.east);
        \draw [->,double,thick] (CE.north) -- (EQ.south);

        \draw [->,double] (CP) -| (C);
        \draw [->,double] (C) |- (CP1);
    \end{scope}
\end{tikzpicture}

\end{document}

    \caption{Block diagram of the OFDM system}
    \label{diagram}
\end{figure}

Principles of operation for each block:
\begin{itemize}
    \item S2P (Serial to Parallel): Converts serial data into parallel data, splitting high-speed bit streams into K lower-speed bit streams, where K is the number of subcarrier waves in the system.
    \item Mapping: QAM modulation to map pairs of bits into complex-valued constellation symbols according to the mapping\_table.
    \item IFFT(Inverse Fast Fourier Transform): Performs a fast implementation of the Inverse Discrete Fourier Transform, transforming signals from the time domain to the frequency domain, creating orthogonal subcarrier waves.
    \item P2S (Parallel to Serial): Converts parallel data back to serial, returning the signal stream to its original continuous form for transmission.
    \item Add CP:  This operation concatenates a copy of the last CP samples of the OFDM time domain signal to the beginning. This way, a cyclic extension is achieved.
    \item Channel: The wireless channel between transmitter and receiver. Here, we use a simple two-tap multipath channel.
    \item Remove CP: Remove `CP` from the received signal.
    \item FFT(Fast Fourier Transform): Transforming signals from the frequency domain to the time domain.
    \item Channel Estimate: Based on pilot signals, the receiver estimates the transmission channel using estimation algorithms.
    \item Equalize: For each subcarrier, the influence of the channel is removed such that we get the clear (only noisy) constellation symbols back.
    \item Demapping: Transform the constellation points to the pairs of bits according to the demapping\_table. The demapping table is simply the inverse mapping of the mapping\_table.
\end{itemize}

\newpage